\nonstopmode{}
\documentclass[a4paper]{book}
\usepackage[times,inconsolata,hyper]{Rd}
\usepackage{makeidx}
\usepackage[utf8,latin1]{inputenc}
% \usepackage{graphicx} % @USE GRAPHICX@
\makeindex{}
\begin{document}
\chapter*{}
\begin{center}
{\textbf{\huge Package `sbfc'}}
\par\bigskip{\large \today}
\end{center}
\begin{description}
\raggedright{}
\item[Type]\AsIs{Package}
\item[Title]\AsIs{Selective Bayesian Forest Classifier}
\item[Version]\AsIs{1.0}
\item[Date]\AsIs{2015-11-27}
\item[Author]\AsIs{Viktoriya Krakovna}
\item[Maintainer]\AsIs{Viktoriya Krakovna }\email{vkrakovna@gmail.com}\AsIs{}
\item[URL]\AsIs{}\url{http://github.com/vkrakovna/sbfc}\AsIs{}
\item[BugReports]\AsIs{}\url{http://github.com/vkrakovna/sbfc/issues}\AsIs{}
\item[Description]\AsIs{SBFC is an MCMC algorithm for simultaneous feature selection and
classification. This package allows you to run SBFC, make graphs showing the
selected features and feature interactions, and perform other analysis of the
results. See paper: http://arxiv.org/abs/1506.02371}
\item[License]\AsIs{GPL (>= 2)}
\item[Depends]\AsIs{R (>= 2.10)}
\item[Imports]\AsIs{Rcpp (>= 0.12.2), DiagrammeR, Matrix, discretization}
\item[LinkingTo]\AsIs{Rcpp, RcppArmadillo}
\item[RoxygenNote]\AsIs{5.0.1}
\item[LazyData]\AsIs{true}
\item[NeedsCompilation]\AsIs{yes}
\item[Archs]\AsIs{i386, x64}
\end{description}
\Rdcontents{\R{} topics documented:}
\inputencoding{utf8}
\HeaderA{sbfc-package}{Selective Bayesian Forest Classifier}{sbfc.Rdash.package}
\keyword{package}{sbfc-package}
%
\begin{Description}\relax
SBFC is an MCMC algorithm for simultaneous feature selection and
classification. This package allows you to run SBFC, make graphs showing the
selected features and feature interactions, and perform other analysis of the
results. See paper: http://arxiv.org/abs/1506.02371
\end{Description}
%
\begin{Details}\relax

\Tabular{ll}{
Package: & sbfc\\{}
Type: & Package\\{}
Title: & Selective Bayesian Forest Classifier\\{}
Version: & 1.0\\{}
Date: & 2015-11-27\\{}
Author: & Viktoriya Krakovna\\{}
Maintainer: & Viktoriya Krakovna <vkrakovna@gmail.com>\\{}
URL: & http://github.com/vkrakovna/sbfc\\{}
BugReports: & http://github.com/vkrakovna/sbfc/issues\\{}
Description: & SBFC is an MCMC algorithm for simultaneous feature selection and
classification. This package allows you to run SBFC, make graphs showing the
selected features and feature interactions, and perform other analysis of the
results. See paper: http://arxiv.org/abs/1506.02371\\{}
License: & GPL (>= 2)\\{}
Depends: & R (>= 2.10)\\{}
Imports: & 
Rcpp (>= 0.12.2),
DiagrammeR,
Matrix,
discretization\\{}
LinkingTo: & Rcpp, RcppArmadillo\\{}
RoxygenNote: & 5.0.1\\{}
LazyData: & true\\{}
}

Index of help topics:
\begin{alltt}
chess                   Chess End-Game - King+Rook versus King+Pawn on
                        a7
corral                  Corral: synthetic data with correlated
                        attributes
corral_augmented        Augmented corral: synthetic data with
                        correlated attributes augmented with noise
                        features
data_disc               Data set discretization and formatting for SBFC
                        algorithm.
edge_density_plot       Plots the density of edges in a given group
                        over the MCMC iterations
heart                   Heart disease outcomes given health attributes
logposterior_plot       Log posterior plot
madelon                 Madelon: synthetic data set from NIPS 2003
                        feature selection challenge
sbfc                    Selective Bayesian Forest Classifier algorithm
sbfc-package            Selective Bayesian Forest Classifier
sbfc_graph              SBFC graph
signal_size_plot        Trace plot of Group 1 size
signal_var_proportion   Signal variable proportion
\end{alltt}

Run the SBFC algorithm on a data set using the \code{sbfc} function.
Make SBFC graphs based on the MCMC samples using the \code{sbfc\_graph} function.
Other analysis, e.g. feature selection plots using \code{signal\_node\_prop} (based on how often each variable appeared in the signal group).
\end{Details}
%
\begin{Author}\relax
Viktoriya Krakovna
Maintainer: Viktoriya Krakovna <vkrakovna@gmail.com>
\end{Author}
\inputencoding{utf8}
\HeaderA{chess}{Chess End-Game - King+Rook versus King+Pawn on a7}{chess}
\keyword{datasets}{chess}
%
\begin{Description}\relax
Outcomes of chess games given the board descriptions. \\{}
Data set from UCI repository. Training and test splits from SGI.
\end{Description}
%
\begin{Usage}
\begin{verbatim}
data(chess)
\end{verbatim}
\end{Usage}
%
\begin{Format}
\begin{description}

\item[\code{TrainX}] A matrix with 2130 rows and 36 columns.
\item[\code{TrainY}] A vector with 2130 rows.
\item[\code{TestX}] A matrix with 1066 rows and 36 columns.
\item[\code{TestY}] A vector with 1066 rows.

\end{description}
\end{Format}
%
\begin{References}\relax
\Rhref{https://archive.ics.uci.edu/ml/datasets/Chess+(King-Rook+vs.+King-Pawn)}{UCI chess data set}

\Rhref{http://www.sgi.com/tech/mlc/db/chess.names}{SGI listing for chess data set}
\end{References}
\inputencoding{utf8}
\HeaderA{corral}{Corral: synthetic data with correlated attributes}{corral}
\keyword{datasets}{corral}
%
\begin{Description}\relax
This is an artificial domain where the target concept is (X1\textasciicircum{}X2) V (X3\textasciicircum{}X4). \\{}
Data set by R. Kohavi. Training and test splits from SGI.
\end{Description}
%
\begin{Usage}
\begin{verbatim}
data(corral)
\end{verbatim}
\end{Usage}
%
\begin{Format}
\begin{description}

\item[\code{TrainX}] A matrix with 128 rows and 6 columns.
\item[\code{TrainY}] A vector with 128 rows.

\end{description}
\end{Format}
%
\begin{References}\relax
\Rhref{http://www.sgi.com/tech/mlc/db/corral.names}{SGI listing for corral data set}
\end{References}
\inputencoding{utf8}
\HeaderA{corral\_augmented}{Augmented corral: synthetic data with correlated attributes augmented with noise features}{corral.Rul.augmented}
\keyword{datasets}{corral\_augmented}
%
\begin{Description}\relax
This is an artificial domain where the target concept is (X1\textasciicircum{}X2) V (X3\textasciicircum{}X4). \\{}
Data set by R. Kohavi. Training and test splits from SGI. \\{}
Noise features added by V. Krakovna by shuffling copies of real features.\\{}
The SBFC paper uses subsets of this data set with the first 100 and 1000 features.
\end{Description}
%
\begin{Usage}
\begin{verbatim}
data(corral_augmented)
\end{verbatim}
\end{Usage}
%
\begin{Format}
\begin{description}

\item[\code{TrainX}] A matrix with 128 rows and 10000 columns.
\item[\code{TrainY}] A vector with 128 rows.

\end{description}
\end{Format}
%
\begin{References}\relax
\Rhref{http://www.sgi.com/tech/mlc/db/corral.names}{SGI listing for corral data set}

\Rhref{arxiv.org/abs/1506.02371}{SBFC paper describing augmentation of corral data set}
\end{References}
\inputencoding{utf8}
\HeaderA{data\_disc}{Data set discretization and formatting for SBFC algorithm.}{data.Rul.disc}
%
\begin{Description}\relax
Removes rows containing missing data, and discretizes the data set using Minimum Description Length Partitioning (MDLP).
\end{Description}
%
\begin{Usage}
\begin{verbatim}
data_disc(data, n_train = NULL, missing = "?")
\end{verbatim}
\end{Usage}
%
\begin{Arguments}
\begin{ldescription}
\item[\code{data}] Data frame, where the last column must be the class variable.

\item[\code{n\_train}] Number of data frame rows to use as the training set - the rest are used for the test set. If NULL, all rows are used for training, and there is no test set (default=NULL).

\item[\code{missing}] Label that denotes missing values in your data frame (default='?').
\end{ldescription}
\end{Arguments}
%
\begin{Value}
A discretized data set:
\begin{description}
     
\item[\code{TrainX}] Matrix containing the training data.
\item[\code{TrainY}] Vector containing the class labels for the training data.
\item[\code{TestX}] Matrix containing the test data (optional).
\item[\code{TestY}] Vector containing the class labels for the test data (optional).

\end{description}

\end{Value}
\inputencoding{utf8}
\HeaderA{edge\_density\_plot}{Plots the density of edges in a given group over the MCMC iterations}{edge.Rul.density.Rul.plot}
%
\begin{Description}\relax
Plots the edge density for the given group for a range of the MCMC iterations (indicated by \code{start} and \code{end}).
\end{Description}
%
\begin{Usage}
\begin{verbatim}
edge_density_plot(sbfc_result, group, start = 0, end = 1)
\end{verbatim}
\end{Usage}
%
\begin{Arguments}
\begin{ldescription}
\item[\code{sbfc\_result}] An object of class \code{sbfc}.

\item[\code{group}] Which group (0 or 1) to plot edge density for.

\item[\code{start}] The start of the included range of MCMC iterations (default=0, i.e. starting with the first iteration).

\item[\code{end}] The end of the included range of MCMC iterations (default=1, i.e. ending with the last iteration).
\end{ldescription}
\end{Arguments}
\inputencoding{utf8}
\HeaderA{heart}{Heart disease outcomes given health attributes}{heart}
\keyword{datasets}{heart}
%
\begin{Description}\relax
Data set from UCI repository, discretized using the \code{mdlp} package.
\end{Description}
%
\begin{Usage}
\begin{verbatim}
data(heart)
\end{verbatim}
\end{Usage}
%
\begin{Format}
\begin{description}

\item[\code{TrainX}] A matrix with 270 rows and 13 columns.
\item[\code{TrainY}] A vector with 270 rows.

\end{description}
\end{Format}
%
\begin{References}\relax
\Rhref{https://archive.ics.uci.edu/ml/datasets/Statlog+(Heart)}{UCI heart data set}

\Rhref{http://www.sgi.com/tech/mlc/db/heart.names}{SGI listing for heart data set}
\end{References}
\inputencoding{utf8}
\HeaderA{logposterior\_plot}{Log posterior  plot}{logposterior.Rul.plot}
%
\begin{Description}\relax
Plots the log posterior for a range of the MCMC iterations (indicated by \code{start} and \code{end}).
\end{Description}
%
\begin{Usage}
\begin{verbatim}
logposterior_plot(sbfc_result, start = 0, end = 1, type = "trace")
\end{verbatim}
\end{Usage}
%
\begin{Arguments}
\begin{ldescription}
\item[\code{sbfc\_result}] An object of class \code{sbfc}.

\item[\code{start}] The start of the included range of MCMC iterations (default=0, i.e. starting with the first iteration).

\item[\code{end}] The end of the included range of MCMC iterations (default=1, i.e. ending with the last iteration).

\item[\code{type}] Type of plot (either \code{trace} or \code{acf}, default=\code{trace}).
\end{ldescription}
\end{Arguments}
\inputencoding{utf8}
\HeaderA{madelon}{Madelon: synthetic data set from NIPS 2003 feature selection challenge}{madelon}
\keyword{datasets}{madelon}
%
\begin{Description}\relax
This is a two-class classification problem. 
The difficulty is that the problem is multivariate and highly non-linear. 
Of the 500 features, 20 are real features, 480 are noise features. \\{}
Data set from UCI repository, discretized using median cutoffs.
\end{Description}
%
\begin{Usage}
\begin{verbatim}
data(madelon)
\end{verbatim}
\end{Usage}
%
\begin{Format}
\begin{description}

\item[\code{TrainX}] A matrix with 2000 rows and 500 columns.
\item[\code{TrainY}] A vector with 2000 rows.
\item[\code{TestX}] A matrix with 600 rows and 500 columns.
\item[\code{TestY}] A vector with 600 rows.

\end{description}
\end{Format}
%
\begin{References}\relax
\Rhref{https://archive.ics.uci.edu/ml/datasets/Madelon}{UCI madelon data set}
\end{References}
\inputencoding{utf8}
\HeaderA{sbfc}{Selective Bayesian Forest Classifier algorithm}{sbfc}
%
\begin{Description}\relax
Runs the SBFC algorithm on a discretized data set. To discretize your data, use the \code{\LinkA{data\_disc}{data.Rul.disc}} command.
\end{Description}
%
\begin{Usage}
\begin{verbatim}
sbfc(data, nstep = NULL, thin = 50, burnin_denom = 5, cv = T,
  thinoutputs = F)
\end{verbatim}
\end{Usage}
%
\begin{Arguments}
\begin{ldescription}
\item[\code{data}] Discretized data set:
\begin{description}
     
\item[\code{TrainX}] Matrix containing the training data.
\item[\code{TrainY}] Vector containing the class labels for the training data.
\item[\code{TestX}] Matrix containing the test data (optional).
\item[\code{TestY}] Vector containing the class labels for the test data (optional).

\end{description}


\item[\code{nstep}] Number of MCMC steps, default max(10000, 10 * ncol(TrainX)).

\item[\code{thin}] Thinning factor for the MCMC.

\item[\code{burnin\_denom}] Denominator of the fraction of total MCMC steps discarded as burnin (default=5).

\item[\code{cv}] Do cross-validation on the training set (if test set is not provided).

\item[\code{thinoutputs}] Return thinned MCMC outputs (parents, groups, trees, logposterior), rather than all outputs (default=FALSE).
\end{ldescription}
\end{Arguments}
%
\begin{Details}\relax
Data needs to be discretized before running SBFC. \\{}
If the test data matrix TestX is provided, SBFC runs on the entire training set TrainX, and provides predicted class labels for the test data. 
If the test data class vector TestY is provided, the accuracy is computed. 
If the test data matrix TestX is not provided, and cv is set to TRUE, SBFC performs cross-validation on the training data set TrainX, 
and returns predicted classes and accuracy for the training data. \\{}
\end{Details}
%
\begin{Value}
An object of class \code{sbfc}:
\begin{description}
     
\item[\code{accuracy}] Classification accuracy (on the test set if provided, otherwise cross-validation accuracy on training set).
\item[\code{predictions}] Vector of class label predictions (for the test set if provided, otherwise for the training set).
\item[\code{probabilities}] Matrix of class label probabilities (for the test set if provided, otherwise for the training set).
\item[\code{runtime}] Total runtime of the algorithm in seconds.
\item[\code{parents}] Matrix representing the structures sampled by MCMC, where parents[i,j] is the index of the parent of node j at iteration i (0 if node is a root).
\item[\code{groups}] Matrix representing the structures sampled by MCMC, where groups[i,j] indicates which group node j belongs to at iteration j (0 is noise, 1 is signal).
\item[\code{trees}] Matrix representing the structures sampled by MCMC, where trees[i,j] indicates which tree node j belongs to at iteration j.
\item[\code{logposterior}] Vector representing the log posterior at each iteration of the MCMC.
\item[Parameters] \code{nstep}, \code{thin}, \code{burnin\_denom}, \code{cv}, \code{thinoutputs}.

\end{description}

\end{Value}
%
\begin{Examples}
\begin{ExampleCode}
data(chess)
chess_result = sbfc(chess)
data(corral)
corral_result = sbfc(corral, cv=FALSE)
\end{ExampleCode}
\end{Examples}
\inputencoding{utf8}
\HeaderA{sbfc\_graph}{SBFC graph}{sbfc.Rul.graph}
%
\begin{Description}\relax
Plots a sampled MCMC graph or an average of sampled graphs using Graphviz. \\{}
In average graphs, nodes are color-coded according to importance - the proportion of samples where the node appeared in Group 1 (dark-shaded nodes appear more often).
In average graphs, thickness of edges also corresponds to importance: the proportion of samples where the edge appeared.
\end{Description}
%
\begin{Usage}
\begin{verbatim}
sbfc_graph(sbfc_result, iter = 10000, average = T, edge_cutoff = 0.1,
  single_noise_nodes = F, labels = paste0("X", 1:ncol(sbfc_result$parents)),
  save_graphviz_code = F, colorscheme = "blues", ncolors = 7,
  width = NULL, height = NULL)
\end{verbatim}
\end{Usage}
%
\begin{Arguments}
\begin{ldescription}
\item[\code{sbfc\_result}] An object of class \code{sbfc}.

\item[\code{iter}] MCMC iteration of the sampled graph to plot, if \code{average=F} (default=10000).

\item[\code{average}] Plot an average of sampled MCMC graphs (default=TRUE).

\item[\code{edge\_cutoff}] The average graph includes edges that appear in at least this fraction of the sampled graphs, if \code{average=T} (default=0.1).

\item[\code{single\_noise\_nodes}] Plot single-node trees that appear in the noise group (Group 0) in at least 80 percent of the samples, which can be numerous for high-dimensional data sets (default=FALSE).

\item[\code{labels}] A vector of node labels (default=\code{c("X1","X2",...)}).

\item[\code{save\_graphviz\_code}] Save the Graphviz source code in a .gv file (default=FALSE).

\item[\code{colorscheme}] \Rhref{http://www.graphviz.org/doc/info/colors.html}{Graphviz color scheme} for the nodes (default="blues").

\item[\code{ncolors}] number of colors in the palette (default=7).

\item[\code{width}] An optional parameter for specifying the width of the resulting graphic in pixels.

\item[\code{height}] An optional parameter for specifying the height of the resulting graphic in pixels.
\end{ldescription}
\end{Arguments}
%
\begin{Examples}
\begin{ExampleCode}
data(madelon)
madelon_result = sbfc(madelon)
sbfc_graph(madelon_result) 
sbfc_graph(madelon_result, average=FALSE, iter=5000) # graph for 5000th iteration
sbfc_graph(madelon_result, single_noise_nodes=TRUE) # makes a wide graph with 480 single nodes

data(heart)
heart_result = sbfc(heart)
heart_labels = c("Age", "Sex", "Chest\nPain", "Rest\nBlood\nPressure", "Cholesterol", 
"Blood\nSugar", "Rest\nECG", "Max\nHeart\nRate", "Angina", "ST\nDepression", "ST\nSlope",
"Fluoroscopy\nColored\nVessels", "Thalassemia")
sbfc_graph(heart_result, labels=heart_labels, width=700)
\end{ExampleCode}
\end{Examples}
\inputencoding{utf8}
\HeaderA{signal\_size\_plot}{Trace plot of Group 1 size}{signal.Rul.size.Rul.plot}
%
\begin{Description}\relax
Plots the Group 1 size for a range of the MCMC iterations (indicated by \code{start} and \code{end}).
\end{Description}
%
\begin{Usage}
\begin{verbatim}
signal_size_plot(sbfc_result, start = 0, end = 1, samples = F)
\end{verbatim}
\end{Usage}
%
\begin{Arguments}
\begin{ldescription}
\item[\code{sbfc\_result}] An object of class \code{sbfc}.

\item[\code{start}] The start of the included range of MCMC iterations (default=0, i.e. starting with the first iteration).

\item[\code{end}] The end of the included range of MCMC iterations (default=1, i.e. ending with the last iteration).

\item[\code{samples}] Calculate signal group size based on sampled MCMC graphs after burn-in and thinning,
rather than graphs from all iterations (default=FALSE).
\end{ldescription}
\end{Arguments}
\inputencoding{utf8}
\HeaderA{signal\_var\_proportion}{Signal variable proportion}{signal.Rul.var.Rul.proportion}
%
\begin{Description}\relax
For each variable, computes the proportion of the samples in which this variable is in the signal group (Group 1). 
Plots the top \code{nvars} variables in decreasing order of signal proportion.
\end{Description}
%
\begin{Usage}
\begin{verbatim}
signal_var_proportion(sbfc_result, nvars = 10, samples = F,
  label_size = 1)
\end{verbatim}
\end{Usage}
%
\begin{Arguments}
\begin{ldescription}
\item[\code{sbfc\_result}] An object of class \code{sbfc}.

\item[\code{nvars}] Number of top signal variables to include in the plot (default=10).

\item[\code{samples}] Calculate signal variable proportion based on sampled MCMC graphs after burn-in and thinning,
rather than graphs from all iterations (default=FALSE).

\item[\code{label\_size}] Size of variable labels on the X-axis (default=1).
\end{ldescription}
\end{Arguments}
%
\begin{Value}
Signal proportion for the top \code{nvars} variables in decreasing order.
\end{Value}
\printindex{}
\end{document}
